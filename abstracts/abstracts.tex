\documentclass[10pt,a4paper]{article}
\usepackage[utf8]{inputenc}
\usepackage{amsmath}
\usepackage{amsfonts}
\usepackage{amssymb}
\usepackage{lmodern}
\usepackage{libertine}
\usepackage{glossaries}
\usepackage{xspace}

\makeglossaries
\newacronym[nonumberlist]{msms}{MS/MS}{tandem mass-spectrometry}
\newacronym[nonumberlist]{cid}{CID}{collision induced dissociation}
\newacronym[nonumberlist]{hcd}{HCD}{higher-energy collisional dissociation}
\newacronym[nonumberlist]{omt}{\protect\textit{O}-MT}{\protect\textit{O}-methyl transferase}
\newacronym[symbol=LC-MS/MS, nonumberlist]{lcmsms}{LC-MS/MS}{liquid chromatography-tandem mass spectrometry}
\newacronym[nonumberlist,  first={catechol \protect\textit{O}-methyl transferase (COMT, EC 2.1.1.6)}]{mcomt}{COMT}{catechol \protect\textit{O}-methyl transferase}
\newacronym[nonumberlist]{comt}{COMT}{catechol \protect\textit{O}-methyl transferase}
\newacronym[nonumberlist]{somt}{SOMT-2}{soy \protect\textit{O}-methyl transferase}

\newglossaryentry{pfomt}{
	type        = \acronymtype,
	name        = PFOMT,
	text        = PFOMT,
        nonumberlist,
	first       = phenylpropanoid and flavonoid \protect\textit{O}-methyl transferase (PFOMT),
%	see         = [see also]{itc},
	description = \glslink{pfomtg}{phenylpropanoid and flavonoid \protect\textit{O}-methyl transferase}
}

\newcommand{\pfomt}{\gls{pfomt}\xspace}
\newcommand{\somt}{\gls{somt}\xspace}
\begin{document}

 \section*{English Abstract} 

The present study outlines the useability of two plant \glspl{omt}, \pfomt and \somt, of classes I and II for the biocatalytic methylation of common structural motifs encountered throughout the group of in plant polyphenolic compounds.

\pfomt was biophysically characterized through \glsentrydesc{itc} and the solution of a novel crystal structure of its \textit{apo}-form.

 
  \textit{In vivo} studies using \somt showed its capability to methylate flavonoids and stilbenes at the 4'-position. 
 
  It was demonstrated that the activity of class I plant \gls{omt}, \pfomt, could be modulated by pH and magnesium concentration to achieve previously unobserved methylations of non-catecholic moieties. 
 
  Addtionally, a systematic grid of 15 flavonoids with different substitutions at the B-ring was characterized by \gls{msms} studies through \gls{hcd} and \gls{cid}.

\clearpage

\section*{Deutsche Zusammenfassung}

 Die vorliegende Arbeit umreißt die Verwertbarkeit zweier pflanzlicher \textit{O}-Methyl\-trans\-fer\-asen (\glspl{omt}), \pfomt und \somt, der Klassen I and II für die biokatalytische Methylierung in polyphenolischen Verbindungen verbreiteter Strukturmotive. 
 
 Die \pfomt wurde mithilfe der isothermen Titrationskalorimetrie sowie der Lösung einer Kristallstruktur des \textit{apo}-Enzyms biophysikalisch charakterisiert. 
 
 \textit{In vivo} Studien der \somt zeigten die Methylierung von Flavonoiden und Stilbenen an der 4'-Position. 
 
  Es wurde gezeigt, dass die Aktivität der \pfomt durch Variation von pH-Wert und Magnesium so weit moduliert werden kann, dass vorher unbeschriebene Methylierungen von nicht-catecholen Motiven möglich sind. 
 
 Zuletzt wurden fünfzehn Flavonoide mit unterschiedlichen Substitutionsmustern am B-Ring mittels Tandem-Massen\-spek\-tro\-metrie (MS/MS) durch \textit{collision induced dissociation} (\gls{hcd}) und \textit{higher-energy collisional dissociation} (\gls{cid}) charakterisiert.   




\end{document}